\documentclass{Z}
%% need no \usepackage{Sweave}

%% Symbols
\newcommand{\darrow}{\stackrel{\mbox{\tiny \textnormal{d}}}{\longrightarrow}}

\author{Achim Zeileis\\Wirtschaftsuniversit\"at Wien}
\Plainauthor{Achim Zeileis}

\title{Object-oriented Computation of Sandwich Estimators}

\Keywords{covariance matrix estimators, estimating functions, object orientation, \proglang{R}}
\Plainkeywords{covariance matrix estimators, estimating functions, object orientation, R}

\Abstract{
Sandwich covariance matrix estimators are a popular tool in applied regression
modeling for performing inference that is robust to certain types of model misspecification.
Suitable implementations are available in the \proglang{R} system for statistical
computing for certain model fitting functions only (in particular \code{lm()}), but
not for other standard regression functions, such as \code{glm()}, \code{nls()}, or \code{survreg()}.

Therefore, conceptual tools and their translation to computational tools in the package
\pkg{sandwich} are discussed, enabling the computation of sandwich estimators in general
parametric models. Object orientation can be achieved by providing a few extractor
functions---most importantly for the empirical estimating functions---from which various
types of sandwich estimators can be computed.
}

\begin{document}


%\VignetteIndexEntry{Object-oriented Computation of Sandwich Estimators}
%\VignetteDepends{sandwich,zoo}
%\VignetteKeywords{covariance matrix estimators, estimating functions, object orientation, R}
%\VignettePackage{sandwich}


\section{Introduction} \label{sec:intro}

A popular approach to applied parametric regression modeling is to derive estimates
of the unknown parameters via a set of estimating functions (including least squares
and maximum likelihood scores). Inference for these models is typically based on a
central limit theorem in which the covariance matrix is of a sandwich type: a slice
of meat between two slices of bread, pictorially speaking. Employing estimators
for the covariance matrix based on this sandwich form can make inference for the parameters
more robust against certain model misspecifications (provided the estimating
functions still hold and yield consistent estimates). Therefore, sandwich estimators
such as heteroskedasticy consistent (HC) estimators for cross-section data and
heteroskedasitcity and autocorrelation consistent (HAC) estimators for time-series data 
are commonly used in applied regression, in particular in linear regression models.

\cite{hac:Zeileis:2004a} discusses a set of computational tools provided by the
\pkg{sandwich} package for the \proglang{R} system for statistical computing \citep{hac:R:2006}
which allows for computing HC and HAC estimators in linear regression models fitted 
by \code{lm()}. Here, we set out where the discussion of \cite{hac:Zeileis:2004a} ends
and generalize the tools from linear to general parametric models fitted by estimating
functions. This generalization is achieved by providing an object-oriented implementation
for the building blocks of the sandwich that rely only on a small set of extractor
functions for fitted model objects. The most important of these is a method for
extracting the empirical estimating functions---based on this a wide variety of 
meat fillings for sandwiches is provided.

The paper is organized is follows: Section~\ref{sec:model} discusses the model frame
and reviews some of the underlying theory. Section~\ref{sec:R} presents some existing
\proglang{R} infrastructure which can be re-used for the computation of sandwich 
covariance matrices in Section~\ref{sec:vcov}. Section~\ref{sec:illustrations} gives
a brief illustration of the computational tools before Section~\ref{sec:summary}
concludes the paper.


{
\section{Model frame} \label{sec:model}
\nopagebreak 

To fix notations, let us assume we have data in a regression setup, i.e., 
$(y_i, x_i)$ for $i = 1, \dots, n$, that follow some distribution that is 
controlled by a $k$-dimensional parameter vector $\theta$. In many situations,
an estimating function $\psi(\cdot)$ is available for this type of models
such that $\E[\psi(y, x, \theta)] = 0$. Then, under certain weak regularity
conditions \citep[see e.g.,][]{hac:White:1994}, 
$\theta$ can be estimated using an M-estimator $\hat \theta$ implicitely defined as
  \begin{equation} \label{eq:estfun}
    \sum_{i = 1}^n \psi(y_i, x_i, \hat \theta) \quad = \quad 0.
  \end{equation}
This includes in particular maximum likelihood (ML) and ordinary and nonlinear least
squares (OLS and NLS) estimation, where the estimating function $\psi(\cdot)$ is
the derivative of an objective function $\Psi(\cdot)$:
  \begin{equation} \label{eq:score}
    \psi(y, x, \theta) \quad = \quad \frac{\partial \Psi(y, x, \theta)}{\partial \theta}.
  \end{equation}
}

Inference about $\theta$ is then typically performed relying on a central
limit theorem (CLT) of type
  \begin{equation} \label{eq:clt}
    \sqrt{n} \, (\hat \theta - \theta) \quad \darrow \quad N(0, S(\theta)),
  \end{equation}
where $\darrow$ denotes convergence in distribution. For the covariance matrix
$S(\theta)$, a sandwich formula can be given
\begin{eqnarray} \label{eq:sandwich}
  S(\theta) & = & B(\theta) \, M(\theta) \, B(\theta) \\  \label{eq:bread}
  B(\theta) & = & \left( \E[ - \psi'(y, x, \theta) ] \right)^{-1} \\  \label{obj}
  M(\theta) & = & \VAR[ \psi(y, x, \theta) ]
\end{eqnarray}
see Theorem~6.10 in \cite{hac:White:1994}, Chapter~5 in \cite{hac:Cameron+Trivedi:2005},
or \cite{hac:Stefanski+Boos:2002} for further details.
The ``meat'' of the sandwich $M(\theta)$ is the variance of the estimating
function and the ``bread'' is the inverse of the expectation of its first derivative $\psi'$
(again with respect to $\theta$). Note that we use the more evocative names $S$,
$B$ and $M$ instead of the more conventional notation $V(\theta) = A(\theta)^{-1} B(\theta)
A(\theta)^{-1}$.

In correctly specified models estimated by ML (or OLS and NLS with homoskedastic errors),
this sandwich expression for $S(\theta)$ can be simplified because $M(\theta) = B(\theta)^{-1}$,
corresponding to the Fisher information matrix. Hence, the variance $S(\theta)$ in the CLT from
Equation~\ref{eq:clt} is typically estimated by an empirical version of $B(\theta)$.
However, more robust covariance matrices can be obtained by employing estimates
for $M(\theta)$ that are consistent under weaker assumptions
\citep[see e.g.,][]{hac:Lumley+Heagerty:1999}
and plugging these into the sandwich formula for $S(\theta)$ from Equation~\ref{eq:sandwich}.
Robustness can be achieved with respect to various types of misspecification, e.g.,
heteroskedasticity---however, consistency of $\hat \theta$ has to be assured, which implies that
at least the estimating functions have to be correctly specified.

Many of the models of interest to us, provide some more structure: the objective function
$\Psi(y, x, \theta)$ depends on $x$ and $\theta$ in a special way, namely it does only
depend on the univariate linear predictor $\eta = x^\top \theta$. Then, the estimating function is of type
\begin{equation} \label{eq:estfunHC}
  \psi(y, x, \theta)
    \quad = \quad \frac{\partial \Psi}{\partial \eta} \cdot \frac{\partial \eta}{\partial \theta}
    \quad = \quad \frac{\partial \Psi}{\partial \eta} \cdot x.
\end{equation}
The partial derivative $r(y, \eta) = \partial \Psi(y, \eta) / \partial \eta$ is in some models
also called ``working residual'' corresponding to the usual residuals in linear regression models.
In such linear-predictor-based models, the meat of the sandwich can also be sloppily written as
\begin{equation} \label{eq:objHC}
  M(\theta) \quad = \quad x \, \VAR[ r(y, x^\top \theta) ] \, x^\top.
\end{equation}
Whereas employing this structure for computing HC covariance matrix estimates is
well-established practice for linear regression
models \citep[see][among others]{hac:MacKinnon+White:1985,hac:Long+Ervin:2000},
it is less commonly applied in other regression models such as GLMs.


\section[Existing R infrastructure]{Existing \proglang{R} infrastructure} \label{sec:R}

To make use of the theory outlined in the previous section, some computational infrastructure
is required translating the conceptual to computational tools. \proglang{R} comes with
a multitude of model-fitting functions that compute estimates $\hat \theta$ and can
be seen as special cases of the framework above. They are typically accompanied by extractor
and summary methods providing inference based on the CLT from Equation~\ref{eq:clt}. For extracting
the estimated parameter vector $\hat \theta$ and some estimate of the covariance matrix $S(\theta)$,
there are usually a \code{coef()} and a \code{vcov()} method, respectively. Based on these estimates,
inference can typically be performed by the \code{summary()} and \code{anova()} methods.
By convention, the \code{summary()} method performs partial $t$ or $z$ tests and the
\code{anova()} method performs $F$ or $\chi^2$ tests for nested models. The covariance estimate
used in these tests (and returned by \code{vcov()}) usually relies on the assumption of correctly
specified models and hence is simply an empirical version of the bread $B(\theta)$ only
(divided by $n$).

For extending these tools to inference based on sandwich covariance matrix estimators, two
things are needed: 1.~generalizations of \code{vcov()} that enable computations of sandwich
estimates, 2.~inference functions corresponding to the \code{summary()} and \code{anova()}
methods which allow other covariance matrices to be plugged in. As for the latter, the package
\pkg{lmtest} \citep{hac:Zeileis+Hothorn:2002} provides \code{coeftest()} and \code{waldtest()} and
\pkg{car} \citep{hac:Fox:2002} provides \code{linear.hypothesis()}---all of these can perform
model comparisons in rather general parametric models, employing user-specified covariance
matrices. As for the former, only specialized solutions of sandwich covariances matrices
are currently available in \proglang{R} packages, e.g., HAC estimators for linear models in
previous versions of \pkg{sandwich} and HC estimators for linear models in \pkg{car} and
\pkg{sandwich}. Therefore, we aim at providing a tool kit
for plugging together sandwich matrices (including HC and HAC estimators and potentially others)
in general parametric models, re-using the functionality that is already provided.


\section{Covariance matrix estimators} \label{sec:vcov}

In the following, the conceptual tools outlined in Section~\ref{sec:model} are translated
to computational tools preserving their flexibility through the use of the estimating 
functions framework and re-using the computational infrastructure that is already available
in \proglang{R}. Separate methods are suggested for computing estimates for the bread
$B(\theta)$ and the meat $M(\theta)$, along with some convenience functions and wrapper
interfaces that build sandwiches from bread and meat.

\subsection{The bread}

Estimating the bread $B(\theta)$ is usually relatively easy and the most popular estimate
is the Hessian, i.e., the mean crossproduct of the derivative of the estimating function
evaluated at the data and estimated parameters:
\begin{equation} \label{eq:Bhat}
  \hat B \quad = \quad \left( \frac{1}{n} \sum_{i = 1}^n - \psi'(y_i, x_i, \hat \theta) \right)^{-1}.
\end{equation}
If an objective function $\Psi(\cdot)$ is used, this is the crossproduct of its
second derivative, hence the name Hessian.

This estimator is what the \code{vcov()} method is typically based on and therefore it can
usually be extracted easily from the fitted model objects, e.g., for ``\code{lm}'' and
``\code{glm}'' it is essentially the \code{cov.unscaled} element returned by the 
\code{summary()} method. To unify the extraction of a suitable estimate for the bread,
\pkg{sandwich} provides a new \code{bread()} generic that should by default return
the bread estimate that is also used in \code{vcov()}. This will usually be the Hessian
estimate, but might also be the expected Hessian \citep[Equation~5.36]{hac:Cameron+Trivedi:2005}
in some models.

The package \pkg{sandwich} provides \code{bread()} methods for ``\code{lm}'' (including ``\code{glm}''
by inheritance), ``\code{coxph}'', ``\code{survreg}'' and ``\code{nls}'' objects. All of them
simply re-use the information provided in the fitted models (or their summaries) and
perform hardly any computations, e.g., for ``\code{lm}'' objects:
\begin{Schunk}
\begin{Sinput}
bread.lm <- function(obj, ...)
{
  so <- summary(obj)
  so$cov.unscaled * as.vector(sum(so$df[1:2]))
}
\end{Sinput}
\end{Schunk}


\subsection{The meat}

While the bread $B(\theta)$ is typically estimated by the Hessian matrix $\hat B$ from
Equation~\ref{eq:Bhat}, various different types of estimators are available for the meat
$M(\theta)$, usually offering certain robustness properties. Most of these estimators are
based on the empirical values of estimating functions. Hence, a natural idea
for object-oriented implementation of such estimators is the following: provide various 
functions that compute different estimators for the meat based on an
\code{estfun()} extractor function that extracts the empirical estimating functions
from a fitted model object. This is what \pkg{sandwich} does: the functions \code{meat()},
\code{meatHAC()} and \code{meatHC()} compute outer product, HAC and HC estimators for
$M(\theta)$, respectively, relying on the existence of an \code{estfun()} method (and potentially
a few other methods). Their design is described in the following.

\subsubsection{Estimating functions}

Whereas (different types of) residuals are typically available as discrepancy measure for
a model fit via the \code{residuals()} method, the empirical values of the estimating functions
$\psi(y_i, x_i, \hat \theta)$ are often not readily implemented in \proglang{R}. Hence,
\pkg{sandwich} provides a new \code{estfun()} generic whose methods should return an
$n \times k$ matrix with the empirical estimating functions:
 \[ \left( \begin{array}{c} \psi(y_1, x_1, \hat \theta) \\ \vdots \\ \psi(y_n, x_n, \hat \theta)
    \end{array} \right). \]
Suitable methods are provided for ``\code{lm}'', ``\code{glm}'', ``\code{rlm}'', ``\code{nls}'',
``\code{survreg}'' and ``\code{coxph}'' objects. Usually, these can easily re-use existing
methods, in particular \code{residuals()} and \code{model.matrix()} if the model is of
type~(\ref{eq:estfunHC}). As a simple example, the most important steps of the ``\code{lm}''
method are
\begin{Schunk}
\begin{Sinput}
estfun.lm <- function (obj, ...) 
{
  wts <- weights(obj)
  if(is.null(wts)) wts <- 1
  residuals(obj) * wts * model.matrix(obj)
}
\end{Sinput}
\end{Schunk}

    
\subsubsection{Outer product estimators}

A simple and natural estimator for the meat matrix $M(\theta) = \VAR[ \psi(y, x, \theta)]$
is the outer product of the empirical estimating functions:
\begin{equation} \label{eq:meatOP}
  \hat M \quad = \quad \frac{1}{n} \sum_{i = 1}^n
    \psi(y_i, x_i, \hat \theta) \psi(y_i, x_i, \hat \theta)^\top
\end{equation}
This corresponds to the Eicker-Huber-White estimator \citep{hac:Eicker:1963,hac:Huber:1967,hac:White:1980}
and is sometimes also called outer product of gradients estimator. In practice, a degrees
of freedom adjustment is often used, i.e., the sum is scaled by $n-k$ instead of $n$,
corresponding to the HC1 estimator from \cite{hac:MacKinnon+White:1985}. In non-linear
models this has no theoretical justification, but has been found to have better finite sample
performance in some simulation studies.

In \pkg{sandwich}, these two estimators are provided by the function \code{meat()} which only
relies on the existence of an \code{estfun()} method. A simplified version of the \proglang{R} code is
\begin{Schunk}
\begin{Sinput}
meat <- function(obj, adjust = FALSE, ...) 
{
  psi <- estfun(obj)
  k <- NCOL(psi)
  n <- NROW(psi)
  rval <- crossprod(as.matrix(psi))/n
  if(adjust) rval <- n/(n - k) * rval
  rval
}
\end{Sinput}
\end{Schunk}


\subsubsection{HAC estimators}

More elaborate methods for deriving consistent covariance matrix estimates in the
presence of autocorrelation in time-series data are also available. Such HAC estimators
$\hat M_\mathrm{HAC}$ are based on the weighted empirical autocorrelations of the empirical
estimating functions:
\begin{equation} \label{eq:meatHAC}
  \hat M_\mathrm{HAC} \quad = \quad \frac{1}{n}
  \sum_{i, j = 1}^n w_{|i-j|} \, \psi(y_i, x_i, \hat \theta) \psi(y_j, x_j, \hat \theta)^\top
\end{equation}
where different strategies are available for the choice of the weights $w_\ell$ at lag
$\ell = 0, \dots, {n-1}$ \citep{hac:Andrews:1991,hac:Newey+West:1994,hac:Lumley+Heagerty:1999}.
Again, an additional finite sample adjustment can be applied by multiplication with $n/(n-k)$.

Once a vector of weights is chosen, the computation of $\hat M_\mathrm{HAC}$ in \proglang{R}
is easy, the most important steps are given by
\begin{Schunk}
\begin{Sinput}
meatHAC <- function(obj, weights, ...)
{
  psi <- estfun(obj)
  n <- NROW(psi)

  rval <- 0.5 * crossprod(psi) * weights[1]
  for(i in 2:length(weights))
    rval <- rval + weights[i] * crossprod(psi[1:(n-i+1),], psi[i:n,])
  
  (rval + t(rval))/n
}
\end{Sinput}
\end{Schunk}
The actual function \code{meatHAC()} in \pkg{sandwich} is much more complex as it also 
interfaces different weighting and bandwidth selection functions. The details are the same
compared to \cite{hac:Zeileis:2004a} where the selection of weights had been discussed for
fitted ``\code{lm}'' objects.


\subsubsection{HC estimators}

In addition to the two HC estimators that can be written as outer product
estimators (also called HC0 and HC1), various other HC estimators (usually
called HC2--HC4) have been suggested, in particular for the linear regression
model \citep{hac:MacKinnon+White:1985,hac:Long+Ervin:2000,hac:Cribari-Neto:2004}.
In fact, they can be applied to more general models provided the estimating
function depends on the parameters only through a linear predictor as
described in Equation~\ref{eq:estfunHC}. Then, the meat matrix $M(\theta)$ is
of type (\ref{eq:objHC}) which naturally leads to HC estimators of the form
$\hat M_\mathrm{HC} = 1/n \, X^\top \hat \Omega X$, where $X$
is the regressor matrix and $\hat \Omega$ is a diagonal matrix estimating the variance of $r(y, \eta)$.
Various functions $\omega(\cdot)$ have been suggested that derive estimates of
the variances from the observed working residuals
$(r(y_1, x_1^\top \hat \theta), \dots, r(y_n, x_n^\top \hat \theta))^\top$---possibly 
also depending on
the hat values and the degrees of freedom. Thus, the HC estimators are of the form
\begin{equation} \label{eq:meatHC}
\hat M_\mathrm{HC} \quad = \quad \frac{1}{n} X^\top \left( \begin{array}{ccc} 
  \omega(r(y_1, x_1^\top \theta)) & \cdots & 0 \\
  \vdots & \ddots & \vdots \\
  0 & \cdots & \omega(r(y, x^\top \theta))
  \end{array} \right) X.
\end{equation}

To transfer these tools into software in the function \code{meatHC()}, we need infrastructure
for three elements in Equation~\ref{eq:meatHC}: 1.~the model matrix $X$, 2.~the function $\omega(\cdot)$,
and 3.~the empirical working residuals $r(y_i, x_i^\top \hat \theta)$. As for 1, the model matrix $X$ can
easily be accessed via the \code{model.matrix()} method. Concerning 2, the specification of $\omega(\cdot)$
is discussed in detail in \cite{hac:Zeileis:2004a}. Hence, we omit the details here and only assume
that we have either a vector \code{omega} of diagonal elements or a function \code{omega} that
computes the diagonal elements from the residuals, diagonal values of the hat matrix (provided
by the \code{hatvalues()} method) and the degrees of freedom $n-k$.
For 3, the working residuals, some fitted model classes provide infrastructure in their \code{residuals()}
method. However, there is no unified interface available for this and instead of setting up
a new separate generic, it is also possible to recover this information from the estimating function.
As $\psi(y_i, x_i, \hat \theta) = r(y_i, x_i^\top \hat \theta) \cdot x_i$, we can simply 
divide the empirical estimating function by $x_i$ to obtain the working residual.

Based on these functions, all necessary information can be extracted from fitted model
objects and a condensed version of \code{meatHC()} can then be written as
\begin{Schunk}
\begin{Sinput}
meatHC <- function(obj, omega, ...)
{
  X <- model.matrix(obj)
  res <- rowMeans(estfun(obj)/X, na.rm = TRUE)
  diaghat <- hatvalues(obj)
  df <- NROW(X) - NCOL(X)  
  
  if(is.function(omega)) omega <- omega(res, diaghat, df)
  rval <- sqrt(omega) * X
  
  crossprod(rval)/NROW(X)
}
\end{Sinput}
\end{Schunk}

\subsection{The sandwich}

Based on the building blocks described in the previous sections, computing a
sandwich estimate from a fitted model object is easy: the
function \code{sandwich()} computes an estimate (by default the Eicker-Huber-White
outer product estimate) for $1/n \, S(\theta)$ via
\begin{Schunk}
\begin{Sinput}
sandwich <- function(obj, bread. = bread, meat. = meat, ...)
{
  if(is.function(bread.)) bread. <- bread.(obj)
  if(is.function(meat.)) meat. <- meat.(obj, ...)
  1/NROW(estfun(obj)) * (bread. %*% meat. %*% bread.)
}
\end{Sinput}
\end{Schunk}
For computing other estimates, the argument \code{meat.} could also be set to
\code{meatHAC} or \code{meatHC}. 

Therefore, all that a use\proglang{R}/develope\proglang{R} would have to do to make a
new class of fitted models, ``\code{foo}'' say, fit for this framework is: 
provide an \code{estfun()} method \code{estfun.}\emph{foo}\code{()}
and a \code{bread()} method \code{bread.}\emph{foo}\code{()}. See also Figure~\ref{fig:sandwich}.

Only for HC estimators (other than HC0 and HC1 which are available via \code{meat()}),
it has to be assured in addtion that 
\begin{itemize}
  \item the model only depends on a linear predictor (this cannot be easily
        checked by the software, but has to be done by the user),
  \item the model matrix $X$ is available via a \code{model.matrix.}\emph{foo}\code{()} method,
  \item a \code{hatvalues.}\emph{foo}\code{()} method exists (for HC2--HC4).
\end{itemize}

For both, HAC and HC estimators, the complexity of the meat functions was reduced for
exposition in the paper: choosing the \code{weights} in \code{meatHAC} and the diagonal elements
\code{omega} in \code{meatHC} can be controlled by a number of further arguments. To make
these explicit for the user, wrapper functions \code{vcovHAC()} and \code{vcovHC()} are provided
in \pkg{sandwich} which work as advertised in \cite{hac:Zeileis:2004a} and are the recommended
interfaces for computing HAC and HC estimators, respectively. Furthermore, the convenience
interfaces \code{kernHAC()}, \code{NeweyWest()} and \code{weave()} setting the right defaults for
\citep{hac:Andrews:1991}, \cite{hac:Newey+West:1994}, and \cite{hac:Lumley+Heagerty:1999}, respectively,
continue to be provided by \pkg{sandwich}.

\setkeys{Gin}{width=.85\textwidth} 
\begin{figure}[tbh]
\begin{center}
\includegraphics{sandwich-OOP-sandwich}
\caption{\label{fig:sandwich} Structure of sandwich estimators}
\end{center}
\end{figure}

\section{Illustrations} \label{sec:illustrations}

%% Applications and illustrations
%%  - linear model in Zeileis (2004)
%%  - generalized linear model
%%  - survreg: tobit regression for data("Fair", package = "Ecdat")
%%  - betareg -> send code to CN and dBS

For briefly illustrating how the sandwich estimators discussed in the
previous sections can easily be applied in practice, we consider an
example from \citet[Section~22.3.6]{hac:Greene:2003} that reproduces
the analysis of extramarital affairs by \citet{hac:Fair:1978}. The data,
famously known as Fair's affairs, is available in the \pkg{Ecdat} package
\citep{hac:Croissant:2005} and provides cross-section information on the number
of extramarital affairs of 601 individuals along with several covariates such
as age (\code{age}), years married (\code{ym}), religiousness (\code{religious}),
occupation (\code{occupation}) and a self-rating of the marriage (\code{rate}).
Table~22.3 in \cite{hac:Greene:2003} provides the parameter estimates and corresponding
standard errors of a tobit model (for the number of affairs) and a probit model
(for infidelity as a binary variable). In \proglang{R}, these models can be
fitted using \code{survreg()} from the \pkg{survival} package \citep{hac:Thernau+Lumley:2006}
and \code{glm()}, respectively:

\begin{Schunk}
\begin{Sinput}
> data("Fair", package = "Ecdat")
> fm_tobit <- survreg(Surv(nbaffairs, nbaffairs > 0, type = "left") ~ 
+     age + ym + religious + occupation + rate, data = Fair, dist = "gaussian")
> fm_probit <- glm(I(nbaffairs > 0) ~ age + ym + religious + occupation + 
+     rate, data = Fair, family = binomial(link = probit))
\end{Sinput}
\end{Schunk}

Using \code{coeftest()} from \pkg{lmtest} \citep{hac:Zeileis+Hothorn:2002},
we can produce the usual summary based on the standard errors as computed by
\code{vcov()} \citep[which reproduces the results in][]{hac:Greene:2003}
and compare them to the HC standard errors provided by \code{sandwich()}.

\begin{Schunk}
\begin{Sinput}
> coeftest(fm_tobit)
\end{Sinput}
\begin{Soutput}
z test of coefficients:

             Estimate Std. Error z value  Pr(>|z|)    
(Intercept)  8.174197   2.741446  2.9817  0.002866 ** 
age         -0.179333   0.079093 -2.2674  0.023368 *  
ym           0.554142   0.134518  4.1195 3.798e-05 ***
religious   -1.686220   0.403752 -4.1764 2.962e-05 ***
occupation   0.326053   0.254425  1.2815  0.200007    
rate        -2.284973   0.407828 -5.6028 2.109e-08 ***
Log(scale)   2.109859   0.067098 31.4444 < 2.2e-16 ***
---
Signif. codes:  0 '***' 0.001 '**' 0.01 '*' 0.05 '.' 0.1 ' ' 1 
\end{Soutput}
\begin{Sinput}
> coeftest(fm_tobit, vcov = sandwich)
\end{Sinput}
\begin{Soutput}
z test of coefficients:

             Estimate Std. Error z value  Pr(>|z|)    
(Intercept)  8.174197   3.077933  2.6557  0.007913 ** 
age         -0.179333   0.088915 -2.0169  0.043706 *  
ym           0.554142   0.137162  4.0400 5.344e-05 ***
religious   -1.686220   0.399854 -4.2171 2.475e-05 ***
occupation   0.326053   0.245978  1.3255  0.184993    
rate        -2.284973   0.393479 -5.8071 6.356e-09 ***
Log(scale)   2.109859   0.054837 38.4754 < 2.2e-16 ***
---
Signif. codes:  0 '***' 0.001 '**' 0.01 '*' 0.05 '.' 0.1 ' ' 1 
\end{Soutput}
\end{Schunk}

In this case, the HC standard errors are only slightly different and yield
qualitatively identical results. The picture is similar for the probit model
which leads to the same interpretations, both for the standard and the HC 
estimate.

\begin{Schunk}
\begin{Sinput}
> coeftest(fm_probit)
\end{Sinput}
\begin{Soutput}
z test of coefficients:

             Estimate Std. Error z value  Pr(>|z|)    
(Intercept)  0.976668   0.365375  2.6731 0.0075163 ** 
age         -0.022024   0.010319 -2.1343 0.0328214 *  
ym           0.059901   0.017121  3.4986 0.0004677 ***
religious   -0.183646   0.051715 -3.5511 0.0003836 ***
occupation   0.037513   0.032845  1.1421 0.2533995    
rate        -0.272983   0.052574 -5.1923 2.077e-07 ***
---
Signif. codes:  0 '***' 0.001 '**' 0.01 '*' 0.05 '.' 0.1 ' ' 1 
\end{Soutput}
\begin{Sinput}
> coeftest(fm_probit, vcov = sandwich)
\end{Sinput}
\begin{Soutput}
z test of coefficients:

             Estimate Std. Error z value  Pr(>|z|)    
(Intercept)  0.976668   0.393020  2.4850 0.0129538 *  
age         -0.022024   0.011274 -1.9535 0.0507577 .  
ym           0.059901   0.017556  3.4120 0.0006449 ***
religious   -0.183646   0.053046 -3.4620 0.0005361 ***
occupation   0.037513   0.032922  1.1395 0.2545052    
rate        -0.272983   0.053326 -5.1191  3.07e-07 ***
---
Signif. codes:  0 '***' 0.001 '**' 0.01 '*' 0.05 '.' 0.1 ' ' 1 
\end{Soutput}
\end{Schunk}

See \cite{hac:Greene:2003} for a more detailed discussion of these and other
regression models for Fair's affairs data.

This brief illustration should merely show that for the use\proglang{R}, the
estimators can be easily employed for inference. As discussed in Section~\ref{sec:R},
further inference functions are available that accept covariance matrix estimators
(or estimates) as arguments, e.g., in packages \pkg{lmtest} or \pkg{car}.

\section{Summary} \label{sec:summary}

Object-oriented computational infrastructure in the \proglang{R} package \pkg{sandwich}
for estimating sandwich covariance matrices in a wide class of parametric models is suggested.
Re-using existing building blocks, all a develope\proglang{R} has to provide for adapting
a new fitted model class to the sandwich estimators are methods for extracting a bread
estimator and the empirical estimating functions (and possibly model matrix and hat values).

\section*{Acknowledgements}

The extensions of \pkg{sandwich}, in particular to microeconometric models,
was motivated by the joint work with Christian Kleiber on \cite{hac:Kleiber+Zeileis:2006}.
Furthermore, we would like to thank Henric Nilsson for helpful comments and discussions that
lead to improvement and generalization of the functions in the package.

\bibliography{hac}

\end{document}
